\documentclass{article}
\usepackage[utf8]{inputenc}
\usepackage[italian]{babel}
\usepackage{float}
\usepackage{graphicx}
\usepackage[margin=1.5in]{geometry}
\usepackage{makecell}
\usepackage{svg}
\usepackage{pdfpages}
\usepackage{subfig}

\title{Motivazione della scelta}
\author{Dot Seven (Gruppo 8)}
\date{28 Ottobre 2022}

\begin{document}

\maketitle
\begin{figure}[!h]
    \centering
    \subfloat{{\includegraphics[width=3cm]{../../res/dotseven_t.png} }}%
\end{figure}
\tableofcontents

\newpage


\section{Introduzione}

Con la seguente, il gruppo .7 (Dot Seven) intende comunicare la decisione di prendere in carico il capitolato \textbf{C1}, denominato
\textit{\textbf{CAPTCHA: Umano o Sovrumano?}}, commissionato da Zucchetti S.P.A. Successivamente alla presentazione dei capitolati il gruppo ha effettuato una prima votazione per capire quali fossero i più interessanti e, in seguito, una riunione dove sono stati analizzati più nel dettaglio.

\section{Motivazioni}

\begin{itemize}
    
    \item 
    Durante questa riunione si è deciso di scartare subito il capitolato \textbf{C4} in quanto non risultava tra le prime scelte di nessun membro del gruppo.
    \item
    Per quanto riguarda i capitolati \textbf{C2} e \textbf{C5}, nonostante avessero riscontrato abbastanza consensi tra i vari membri del gruppo, si è deciso di non sceglierli come possibili opzioni,  a causa del fatto che già molti altri gruppi li avevano candidati come possibile scelta.
    \item
    Il capitolato \textbf{C7} è stato scartato a seguito di un’analisi esplorativa, dalla quale è risultato che nessun componente del gruppo conosceva le tecnologie che sarebbero dovute essere impiegate, di conseguenza, le ore necessarie all'apprendimento delle stesse sarebbero state troppo elevate, inoltre l'interesse da parte del gruppo era scarso.
    \item 
    Infine, è stato deciso di effettuare degli incontri conoscitivi e di approfondimento con le aziende proponenti dei capitolati \textbf{C1} e \textbf{C3},  più apprezzati dalla maggioranza del gruppo rispetto ai rimanenti \textbf{C5} e \textbf{C6}, stando ai dati ottenuti in seguito ad una votazione interna.
    L’incontro con Infocert (C3), nonostante il chiarimento di alcuni dubbi, ne ha, contemporaneamente, fatti emergere ulteriori su alcune caratteristiche tecniche del progetto non inizialmente indicate nel documento di presentazione del capitolato. Alla luce di queste scoperte, il gruppo ha deciso di escludere il capitolato.
    
\end{itemize}

\section{Scelta finale}

La scelta finale è ricaduta, dunque, sul capitolato \textbf{C1}, in quanto vede nelle aree di ricerca presentate un settore in forte evoluzione e di interesse strategico anche per un futuro lavorativo, oltre che un'interessante sfida nel concepire un sistema di CAPTCHA che sia un buon compromesso tra accessibilità ed efficacia contro le intelligenze artificiali. Le tecnologie di ML e AI sono sempre più usate, lo studio e lo sviluppo per limitarne l'uso malevolo sono sempre più un’urgenza per la sicurezza del web. Inoltre, il gruppo ha ritenuto interessante la proposta di lavorare con un’azienda così grande ed importante in Italia.
In seguito alle varie consultazioni sia con il proponente sia interne il gruppo ha deciso di candidarsi per il capitolato C1, nell’intenzione di svolgerlo con elevati standard di qualità.

\end{document}
